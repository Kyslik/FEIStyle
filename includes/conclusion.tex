\begin{quote}
\it\uv{V tejto príručke sme sa snažili poskytnúť praktické 
pokyny pre vypracovanie záverečných prác na
Fakulte elektrotechniky a informatiky STU
v~Bratislave.
Obsahuje dôležité informácie o~formálnej stránke dokumentu,
ktoré sú nevyhnutné pre zabezpečenie jasnej štruktúry 
a~správneho formátovania práce.
Naša príručka kladie dôraz na dodržiavanie základných 
pravidiel,
ktoré pomáhajú minimalizovať chyby a~zlepšujú celkovú 
kvalitu predkladaných dokumentov.

Rovnakú pozornosť venujeme aspektom, ako sú formátovanie textu, zarovnávanie, riadkovanie a použitie preddefinovaných štýlov, ktoré prispievajú k~profesionálnemu vzhľadu dokumentu. Dôkladné spracovanie matematických rovníc a grafov je ďalším dôležitým prvkom, ktorý môže ovplyvniť hodnotenie práce. Práca s týmito aspektmi zaisťuje, že obsah je prehľadný a presne odráža zamýšľané myšlienky.

V neposlednom rade sa naša príručka zameriava na význam citovania externých zdrojov, čo je zásadné pre zachovanie integrity a etických štandardov pri písaní akademických textov. Správne citovanie nielenže posilňuje argumentáciu, ale tiež pomáha predchádzať plagiátorstvu, ktoré môže mať vážne následky na akademickú kariéru.

Dodržiavanie týchto pokynov umožňuje vyhnúť sa bežným chybám a~zvyšuje kvalitu záverečných prác, čím sa zlepšuje celkový dojem z akademických schopností. Dobre vypracovaná práca môže viesť k pozitívnemu hodnoteniu a~uznaniu zo strany hodnotiteľov, čo je cieľom každého študenta.

Celkovo je príručka navrhnutá ako cenný nástroj na podporu pri písaní záverečných prác. Poskytnuté informácie a odporúčania prispejú k dosiahnutiu akademických cieľov a pomôžu vypracovať kvalitné a~odborné dokumenty, ktoré budú spĺňať štandardy fakulty.}

(ChatGPT 3, 2024)
\end{quote}

Poznámka autora:
Časť záveru vygenerovala umelá inteligencia 
na základe analýzy obsahu 
verzie dokumentu pre práce písané v MS Worde.
Konkrétne išlo o aplikáciu ChatGPT 3 dňa 29. 9. 2024.
Rukopis umelej inteligencie je z~textu cítiť aj napriek tomu,
že celá komunikácia vedúca k~tejto forme záveru
trvala približne 40 minút.
Parafrázujúc ľudové príslovie konštatujeme,
že umelá inteligencia je možno dobrý sluha,
ale zlý pán.
Používajme ju s~rozumom.

Pripomíname, že takýto spôsob tvorby obsahu je
v~záverečnej práci, ako aj v~iných akademických textoch, 
neprípustný.
Autorské dielo nesmie obsahovať odseky vytvorené
generatívnou umelou inteligenciou.
Preto sme ho formátovali ako citovaný text.
