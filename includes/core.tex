\section{Ukážka glossaries}
Verzia FEIstyle 1.5 používa glossary\footnote{\url{https://www.ctan.org/pkg/glossaries?lang=en}} balík.
\acrfull{cdma} je dlhá skratka naopak \acrshort{gsm} je skratka v krátkej forme.
\section{Recitácia}
Citujem všetky zdroje v \textbf{bibliography.bib}, \cite{t00, t01, t02, t03, kniha, kniha2, kniha3, small, big, cs, koll, kap, tug, knuth, zbornik, prispevok}. \newline Good luck.
\section{Možnosti anonymizácie}
Anonymizácia znamená zmena alebo úprava údajov tak, aby sa podľa nich nedala jednoznačne určiť osoba, ktorej tieto údaje patria \cite{t01}. Existuje niekoľko spôsobov, ktorými môžeme dosiahnuť rôznu úroveň anonymizácie na internete: od mazania cookies súborov po ukončení prehliadania webových stránok až po používanie operačných systémov, ktoré sú na anonymite založené; od bezplatných možností až po komerčné verzie.  
\newline Nasleduje priblíženie niektorých možnosti anonymizácie.

\subsection{Súkromné prehliadanie}
Najpoužívanejšie internetové prehliadače súčasnosti majú v sebe zabudovanú funkcionalitu, ktorá dokáže čiastočne anonymizovať prístup na internet. Táto funkcionalita blokuje ukladanie navštívených stránok do histórie a nezaznamenáva súbory, ktoré sa stiahnu z~internetu. \acrshort{sw} a \acrlong{hw} sú skratky.

\begin{table}[ht]
\caption{Moduly a ich funkcie pri anonymizácii}
\label{modulyVlastnosti}
\begin{center}
\begin{tabular}{p{4cm}|c|c|c|c|c|c|c|c|c|c|c|c|c|c|c}
& \multicolumn{14}{c}%
	 {\textbf{Funkcia}}\\ \hline
&&&& & &\multicolumn{8}{c}%
	 {Modifikácia}\\ 
\textbf{Modul} &\begin{sideways} zobrazenie hlavičky \end{sideways} &\begin{sideways} blokovanie skriptov \end{sideways} &\begin{sideways} zmena IP \end{sideways} & \begin{sideways} zmena lokalizácie \end{sideways} & \begin{sideways} zmazanie/blokovanie cookies \end{sideways} & \begin{sideways} blokovanie trackerov \end{sideways}  & \begin{sideways} popis \end{sideways} & \begin{sideways}používateľský agent\end{sideways} & \begin{sideways} kódové označenie prehliadača \end{sideways} & \begin{sideways} názov prehliadača \end{sideways} & \begin{sideways} verzia prehliadača \end{sideways} & \begin{sideways} platforma \end{sideways} & \begin{sideways} výrobca prehliadača \end{sideways} & \begin{sideways} označenie výrobcu prehliadača \end{sideways} \\ \hline
User agent switcher & & & & & &  & X & X & X & X & X & X & X & X  \\ \hline
Ghostery &  && & & X & X &  &  & & & & & & \\  \hline
Better privacy && &  & & X &  &  &  & & & & & & \\  \hline
Anonymox &  && X & X & X &  & X & X & & & & & & \\  \hline
Modify headers & & &  &  & X &  &  & X &  &  &  & & &  \\  \hline
Request policy & & &  &  & & X  &  &  &  &  &  & & &   \\  \hline
Live HTTP headers & X& &  &  & &  &  &  &  &  &  & & &   \\  \hline
User agent awitcher & & &  &  & &  & X & X &  &  &  & & &   \\  \hline
Header hacker & & &  &  & &  & X & X & X & X & X & X & X & X    \\  \hline
Mod header & & &  &  & &  & X & X & X & X & X & X & X & X    \\  \hline
Script no & &X &  &  & &  &  &  &  &  &  &  &  &     \\  \hline
No script & &X &  &  & &  &  &  &  &  &  &  &  &     \\  \hline
Proxify it & & &X  & X & &  &  &  &  &  &  &  &  &     \\  \hline
I'm not here & & &  & X & &  &  &  &  &  &  &  &  &     \\  \hline
Get edition & &X &X &X &X&X &  &  &  &  &  &  &  &     \\  \hline
\end{tabular}
\end{center}
\end{table}

\begin{table}[ht]
    \centering
    \caption{Príklad tabuľky s použitím balíka \texttt{booktabs}}
    \begin{tabular}{@{}ll@{}} \toprule
         Veľkosť (B) & Typ správy \\ \midrule
         8 & Nejaký typ správy \\ \bottomrule
    \end{tabular}
    \label{tab:my_label}
\end{table}

\subsection{Anonymná sieť}
Anonymná sieť je sieť serverov, medzi ktorými dáta prechádzajú šifrované. V anonymných sieťach dáta prechádzajú z počítača používateľa, odkiaľ bola požiadavka poslaná, cez viaceré proxy smerovače, z ktorých každý správu doplní o smerovanie a zašifruje vlastným kľúčom. Cesta od ...


\subsection{Funkcionalita}
Rozšírenie tiež okrem splnenia špecifikácie malo pre prehľadnosť a overenie funkčnosti zobrazovať údaje, ktoré boli na server odoslané. Zoznam údajov odoslaných na server, sa mal ukladať do krátkodobej histórie, aby nemal používateľ k dispozícií len najnovšie údaje, ale aj údaje odoslané v nejakom časovom období. Nejaky listing z priloh \ref{lst:sublime}.

\subsubsection{Funkcionalita2}
Samozrejmosťou bolo nastavenie zapnutia rozšírenia pri štarte, prípadne interval zmeny odosielaných údajov.

\subsection{Vzhľad}
Dôležitou požiadavkou kladenou na rozšírenie bolo príjemné používateľské rozhranie. Z~tohto dôvodu malo rozšírenie obsahovať zoznam modifikovaných vlastností a tlačidlo pre prístup k nastaveniam rozšírenia v jednoduchej a praktickej forme. Predpokladaný vzhľad je zobrazený na obrázku č. \ref{vzhladobr}.
\begin{figure}[!htbp]
  \centering
  \includegraphics[width=8cm]{img/vzhlad.png}
  \caption{Predpokladaný vzhľad rozšírenia.}
  \label{vzhladobr}
\end{figure}	 
Dôležitou požiadavkou kladenou na rozšírenie bolo príjemné používateľské rozhranie.\cite{t00} Z~tohto dôvodu malo rozšírenie obsahovať zoznam modifikovaných vlastností a tlačidlo pre prístup k nastaveniam rozšírenia v jednoduchej a praktickej forme. Predpokladaný vzhľad je zobrazený na obrázku č. \ref{vzhladobr}.

\begin{algorithm}
\scriptsize
\begin{algorithmic}
 \STATE <text>
 \IF{<condition>} \STATE {<text>} \ELSE \STATE{<text>} \ENDIF
 \IF{<condition>} \STATE {<text>} \ELSIF{<condition>} \STATE{<text>} \ENDIF
 \FOR{<condition>} \STATE {<text>} \ENDFOR
 \FOR{<condition> \TO <condition> } \STATE {<text>} \ENDFOR
 \FORALL{<condition>} \STATE{<text>} \ENDFOR
 \WHILE{<condition>} \STATE{<text>} \ENDWHILE
 \REPEAT \STATE{<text>} \UNTIL{<condition>}
 \LOOP \STATE{<text>} \ENDLOOP
 \REQUIRE <text>
 \ENSURE <text>
 \RETURN <text>
 \PRINT <text>
 \COMMENT{<text>}
 \AND, \OR, \XOR, \NOT, \TO, \TRUE, \FALSE
\end{algorithmic}
\caption{Ukážka príkazov pre algorithmic}  
\label{alg:preview}  
\end{algorithm}

\begin{lstlisting}[
  caption={Ukážka algoritmu},
  label={lst:main-c},
  language=c,
  style=code-listing
]
/* Hello World program */

#include<stdio.h>

struct cpu_info {
    long unsigned utime, ntime, stime, itime;
    long unsigned iowtime, irqtime, sirqtime;
};

main()
{
    printf("Hello World");
}
\end{lstlisting}
