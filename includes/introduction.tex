Písanie záverečnej práce je nevyhnutnou súčasťou štúdia na vysokej škole.
Študenti sa s~touto činnosťou často stretávajú po prvýkrát na konci bakalárskeho štúdia.
Príprava obsiahlejšieho textu je náročná úloha,
ktorá zväčša trvá niekoľko týždňov alebo mesiacov.
Pri finalizácii a~formátovaní do výslednej podoby často panuje neistota a~medzi študentami sa šíria protichodné informácie.
Tento dokument slúži ako odporúčaná šablóna pre všetkých študentov FEI STU.
Môžu ju využiť v~plnej miere, alebo len ako inšpiráciu vytvorenia vlastnej výslednej podoby.
Zároveň obsahuje návody a~rady,
ako sa vysporiadať s~problémami,
ktoré sa môžu pri písaní práce vyskytnúť.

Dokument je vytvorený v systéme na sadzbu textu \LaTeX\ použitím upravenej verzie obľúbenej šablóny
FEIstyle, verzia 2.0.0. Príručka je po formálnej a vizuálnej stránke navrhnutá tak,
aby zodpovedala požiadavkám kladeným na záverečnú prácu vrátane poradia jednotlivých častí.
Niektoré časti sú povinné (napr. obálka, titulný list, abstrakt),
iné sú voliteľné (zoznam obrázkov, tabuliek, poďakovanie).
Odporúčame študentom prečítať si aspoň prvú kapitolu tejto príručky, aby získali istotu pri začatí práce.

Jedným zo zámerov tohto manuálu je,
aby jeho zdrojové súbory slúžili ako šablóna,
do ktorej študenti vložia obsah vlastnej práce.
Okrem definície rozmerov, okrajov, číslovania strán, štýlov a znakov, šablóna \verb|FEIstyle.cls| obsahuje množstvo makier a príkazov na generovanie zoznamov obrázkov, tabuliek, výpisov kódov, algoritmov alebo obsahu.
Zoznam a opis všetkých funkcií šablóny nájdete v~priloženom dokumente \verb|tutorial.pdf|.

Je žiadúce, aby študenti ovládali prácu s \LaTeX-om na pokročilej užívateľskej úrovni.
Je vhodné, ak si pred začiatkom písania preštudujú základné príručky. Veľmi vhodná je kniha
Jiřího Rybyčku \LaTeX\ pro začátečníky z roku 2003 \cite{Rybicka2003Latex}.
Ide síce o~staršiu publikáciu v~českom jazyku,
patrí však k~vyhľadávaným didaktickým textom
a~je oceňovaná všetkými používateľmi \LaTeX-u najmä pri zvládaní prvých krokov.
Prehľadnou a príťažlivou formou je tiež spracovaný tutoriál pre používateľov internetovej služby Overleaf.%
\footnote{\href{https://www.overleaf.com/learn}{\texttt{www.overleaf.com/learn}}}
Je to bezplatný výkonný online kompilátor \LaTeX-u podporovaný mnohými vedeckými vydavateľstvami.

V~texte tejto rozšírenej šablóny nájdeme okrem iného aj poznámky k~jazykovej stránke textu záverečných prác, písaniu matematických vzorcov, vytváraniu a formátovaniu grafov, tabuliek, či výpisu kódov programov.

V závere \ref{sec:formatLanguage}. kapitoly rozoberieme aktuálnu problematiku používania umelej inteligencie pri tvorbe akademických prác.

Citáciám a bibliografickým zdrojom venujeme celú kapitolu \ref{sec:citation},
pretože práca s~externými zdrojmi je pri akademických textoch veľmi dôležitá.
Záverečná práca má byť samostatné dielo študenta, 
ktorý čerpá poznatky z~iných zdrojov,
spracováva ich a~spája do nových súvislostí,
čím vytvára novú kvalitu.
Veríme, že tento dokument študentom pomôže lepšie sa orientovať,
dodá im istotu pri písaní,
zabezpečí kvalitný výsledok, estetický vzhľad
a~hodnotný obsah.
Táto príručka nie je komplexným nástrojom na všetky aspekty záverečnej práce,
no poskytuje pevný základ úspešného písania
a efektívnej práce s~literatúrou. 

Naším cieľom bolo vytvoriť materiál,
ktorý vedie študenta pri hľadaní ďalších zdrojov v~literatúre a na internete,
aby sa vedel orientovať v~terminológii a~získal základný prehľad.
Prajeme úspešné písanie a~dosiahnutie všetkých stanovených cieľov.